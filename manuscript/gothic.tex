\documentclass[12pt]{scrbook}

\usepackage{authblk}
\usepackage[ngerman]{babel}
\usepackage[natbib,notes,backend=bibtex]{biblatex-chicago}
   \bibliography{references}
   \renewcommand*{\nameyeardelim}{,}
   \renewcommand*{\multicitedelim}{\addcomma\space}
\usepackage{booktabs}
\usepackage{csquotes}
\usepackage{enumitem}
\usepackage{float}
\usepackage[top=4cm,bottom=4cm,left=3cm,right=2.5cm]{geometry}
\usepackage[hidelinks]{hyperref}
   \urlstyle{same}
\usepackage[utf8]{inputenc}
\usepackage{longtable}
\usepackage{siunitx}
\usepackage{tabto}
\usepackage{tikz}

\deffootnote{1.5em}{1em}{\makebox[1.5em][l]{\thefootnotemark}}
   \setlength{\skip\footins}{1.5em} 
   \setlength{\footnotesep}{1em}

\NumTabs{14}

\title{Gothic}
\subtitle{}

\author{Alexander Max Bauer}

\date{}


%%%%%%%%%%%
% TITELEI %
%%%%%%%%%%%
\begin{document}
\maketitle


%%%%%%%%%%%%%%%%%%%%%%
% INHALTSVERZEICHNIS %
%%%%%%%%%%%%%%%%%%%%%%
\frontmatter
\tableofcontents


%%%%%%%%%%%
% VORWORT %
%%%%%%%%%%%
\chapter{Vorwort}


%%%%%%%%%%%
% ORPHEUS %
%%%%%%%%%%%
\mainmatter
\chapter{Orpheus – Ein Gefängnis in Planung}
Gothic hieß nicht immer Gothic.
Der erste Name, den das Projekt trug, war Orpheus.
Aus dieser frühen Phase ist eine Handvoll Dokumente erhalten, in denen Mike Hoge erste Ideen festgehalten hat, von denen es am Ende viele, wie wir sehen werden, auch in das fertige Spiel geschafft haben.
Sie lassen sich anhand einiger in ihnen enthaltener Überschriften grob gliedern:

\begin{itemize}
   \item \enquote{Aufträge}\autocite[S.~16--17]{orpheus_b_scribbles}
   \item \enquote{Der rote Faden}\autocite{orpheus_der_rote}
   \item \enquote{Der Werdegang eines SC's}\autocite{orpheus_der_werdegang}
   \item \enquote{Orpheus – Bewegungen}\autocite{orpheus_bewegungen}
   \item \enquote{Orpheus – History}\autocite[S.~2--3]{orpheus_b_scribbles}
   \item \enquote{Orpheus – Interface}\autocite{orpheus_interface}
   \item \enquote{Orpheus – Kreaturen}\autocite[S.~4]{orpheus_b_scribbles}
   \item \enquote{Orpheus – Locations}\autocite[S.~5]{orpheus_b_scribbles}
   \item \enquote{Orpheus – Spielwelt/Gildensystem}\autocite{orpheus_gildensystem}
   \item \enquote{Orpheus – Spielwelt/Gildensystem V2}\autocite{orpheus_gildensystem_v2}
   \item \enquote{Orpheus – Übersicht}\autocite[S.~11--14]{orpheus_b_scribbles}
   \item \enquote{Zusammenfassung zum Orpheus-Konzept}\autocite{orpheus_zusammenfassung}
\end{itemize}

\noindent Daneben gibt es noch einige lose Notizen\footnote{Diese finden sich in \autocite{orpheus_b_scribbles}.} sowie ein Dokument zur Kampfsteuerung,\autocite{orpheus_kampfsteuerung} das -- abgesehen von den Überschriften seiner einzelnen Abschnitte -- keinen eigenen Titel trägt.
Die \enquote{Zusammenfassung zum Orpheus-Konzept} steht vermutlich nicht nur alphabetisch am Ende: Sie stellt das einzige nicht handschriftlich verfasste Schriftstück in der Sammlung dar und ist auf den 12. Dezember 1996 datiert.
Die anderen Dokumente könnten dementsprechend etwas älter sein; der in ihnen enthaltene Entwurf einer Karte wird beispielsweise auf das Vorjahr, also 1995, geschätzt.\autocite{flosha_evolution}


%%%%%%%%%%%%
% APPENDIX %
%%%%%%%%%%%%
\clearpage
\appendix


%%%%%%%%%%%%%%%%
% BIBLIOGRAFIE %
%%%%%%%%%%%%%%%%
\clearpage
\printbibliography

\end{document}
